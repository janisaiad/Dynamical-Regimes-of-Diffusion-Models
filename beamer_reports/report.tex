\documentclass[11pt,a4paper]{article}
\usepackage[utf8]{inputenc}
\usepackage[T1]{fontenc}
\usepackage{amsmath,amssymb,amsfonts}
\usepackage{graphicx}
\usepackage{xcolor}
\usepackage{bm}
\usepackage{physics}

\title{Calcul des Probabilités et Potentiels pour les Régimes Dynamiques des Modèles de Diffusion}
\author{Analyse Mathématique}
\date{\today}

\begin{document}

\maketitle

\section{Introduction}

Dans ce rapport, nous allons analyser le comportement des modèles de diffusion en calculant les probabilités et les potentiels associés à quatre points spécifiques. Cette analyse permettra de mieux comprendre les régimes dynamiques et les transitions critiques qui caractérisent ces modèles, notamment les temps de séparation ($t_s$) et de collapse ($t_c$).

\section{Définition du problème}

Considérons un modèle de mélange gaussien à deux composantes en dimension $d$, avec des centres situés à $\pm \vec{m}$ où $\vec{m} = \tilde{\mu}\sqrt{d}$. Nous souhaitons analyser le comportement de la distribution et du score en quatre points particuliers:
\begin{align}
\vec{x}_1 &= \vec{m} - \vec{\varepsilon} \\
\vec{x}_2 &= \vec{m} + \vec{\varepsilon} \\
\vec{x}_3 &= -\vec{m} - \vec{\varepsilon} \\
\vec{x}_4 &= -\vec{m} + \vec{\varepsilon}
\end{align}

où $\vec{\varepsilon}$ est un petit vecteur, typiquement orthogonal à $\vec{m}$.

\section{Calcul des probabilités}

Pour le modèle de mélange gaussien, la distribution à l'instant $t$ est donnée par:
\begin{align}
P_t(\vec{x}) = \frac{1}{2\sqrt{2\pi\Gamma_t}^d}\left[
e^{-(\vec{x}-\vec{m} e^{-t})^2/(2 \Gamma_t)}
+e^{-(\vec{x}+\vec{m} e^{-t})^2/(2 \Gamma_t)}
\right]
\end{align}

où $\Gamma_t=\sigma^2 e^{-2t} + \Delta_t$ et $\Delta_t = 1-e^{-2t}$.

\subsection{Probabilités aux quatre points}

Calculons maintenant $P_t(\vec{x})$ pour chacun des quatre points:

\subsubsection{Calcul pour $\vec{x}_1 = \vec{m} - \vec{\varepsilon}$}

\begin{align}
P_t(\vec{x}_1) &= \frac{1}{2\sqrt{2\pi\Gamma_t}^d}\left[
e^{-((\vec{m}-\vec{\varepsilon})-\vec{m} e^{-t})^2/(2 \Gamma_t)}
+e^{-((\vec{m}-\vec{\varepsilon})+\vec{m} e^{-t})^2/(2 \Gamma_t)}
\right] \\
&= \frac{1}{2\sqrt{2\pi\Gamma_t}^d}\left[
e^{-(\vec{m}(1-e^{-t})-\vec{\varepsilon})^2/(2 \Gamma_t)}
+e^{-(\vec{m}(1+e^{-t})-\vec{\varepsilon})^2/(2 \Gamma_t)}
\right]
\end{align}

Pour simplifier, notons:
\begin{align}
A_1 &= (\vec{m}(1-e^{-t})-\vec{\varepsilon})^2 \\
B_1 &= (\vec{m}(1+e^{-t})-\vec{\varepsilon})^2
\end{align}

\subsubsection{Calcul pour $\vec{x}_2 = \vec{m} + \vec{\varepsilon}$}

\begin{align}
P_t(\vec{x}_2) &= \frac{1}{2\sqrt{2\pi\Gamma_t}^d}\left[
e^{-((\vec{m}+\vec{\varepsilon})-\vec{m} e^{-t})^2/(2 \Gamma_t)}
+e^{-((\vec{m}+\vec{\varepsilon})+\vec{m} e^{-t})^2/(2 \Gamma_t)}
\right] \\
&= \frac{1}{2\sqrt{2\pi\Gamma_t}^d}\left[
e^{-(\vec{m}(1-e^{-t})+\vec{\varepsilon})^2/(2 \Gamma_t)}
+e^{-(\vec{m}(1+e^{-t})+\vec{\varepsilon})^2/(2 \Gamma_t)}
\right]
\end{align}

Nous notons:
\begin{align}
A_2 &= (\vec{m}(1-e^{-t})+\vec{\varepsilon})^2 \\
B_2 &= (\vec{m}(1+e^{-t})+\vec{\varepsilon})^2
\end{align}

\subsubsection{Calcul pour $\vec{x}_3 = -\vec{m} - \vec{\varepsilon}$}

\begin{align}
P_t(\vec{x}_3) &= \frac{1}{2\sqrt{2\pi\Gamma_t}^d}\left[
e^{-((-\vec{m}-\vec{\varepsilon})-\vec{m} e^{-t})^2/(2 \Gamma_t)}
+e^{-((-\vec{m}-\vec{\varepsilon})+\vec{m} e^{-t})^2/(2 \Gamma_t)}
\right] \\
&= \frac{1}{2\sqrt{2\pi\Gamma_t}^d}\left[
e^{-(-\vec{m}(1+e^{-t})-\vec{\varepsilon})^2/(2 \Gamma_t)}
+e^{-(-\vec{m}(1-e^{-t})-\vec{\varepsilon})^2/(2 \Gamma_t)}
\right]
\end{align}

Nous notons:
\begin{align}
A_3 &= (-\vec{m}(1+e^{-t})-\vec{\varepsilon})^2 \\
B_3 &= (-\vec{m}(1-e^{-t})-\vec{\varepsilon})^2
\end{align}

\subsubsection{Calcul pour $\vec{x}_4 = -\vec{m} + \vec{\varepsilon}$}

\begin{align}
P_t(\vec{x}_4) &= \frac{1}{2\sqrt{2\pi\Gamma_t}^d}\left[
e^{-((-\vec{m}+\vec{\varepsilon})-\vec{m} e^{-t})^2/(2 \Gamma_t)}
+e^{-((-\vec{m}+\vec{\varepsilon})+\vec{m} e^{-t})^2/(2 \Gamma_t)}
\right] \\
&= \frac{1}{2\sqrt{2\pi\Gamma_t}^d}\left[
e^{-(-\vec{m}(1+e^{-t})+\vec{\varepsilon})^2/(2 \Gamma_t)}
+e^{-(-\vec{m}(1-e^{-t})+\vec{\varepsilon})^2/(2 \Gamma_t)}
\right]
\end{align}

Nous notons:
\begin{align}
A_4 &= (-\vec{m}(1+e^{-t})+\vec{\varepsilon})^2 \\
B_4 &= (-\vec{m}(1-e^{-t})+\vec{\varepsilon})^2
\end{align}

\section{Calcul du score}

Le score du modèle de mélange gaussien est donné par:
\begin{align}
S_i(\vec{x}) = -\frac{x_i}{\Gamma_t}+m_i \frac{e^{-t}}{\Gamma_t}\tanh\left( \vec{x} \cdot \vec{m} \, \frac{e^{-t}}{\Gamma_t}\right)
\end{align}

Calculons maintenant le score pour chacun des quatre points.

\subsection{Score au point $\vec{x}_1 = \vec{m} - \vec{\varepsilon}$}

\begin{align}
S_i(\vec{x}_1) &= -\frac{(m_i - \varepsilon_i)}{\Gamma_t}+m_i \frac{e^{-t}}{\Gamma_t}\tanh\left( (\vec{m} - \vec{\varepsilon}) \cdot \vec{m} \, \frac{e^{-t}}{\Gamma_t}\right) \\
&= -\frac{m_i - \varepsilon_i}{\Gamma_t}+m_i \frac{e^{-t}}{\Gamma_t}\tanh\left( (|\vec{m}|^2 - \vec{\varepsilon} \cdot \vec{m}) \, \frac{e^{-t}}{\Gamma_t}\right)
\end{align}

Si $\vec{\varepsilon}$ est orthogonal à $\vec{m}$, alors $\vec{\varepsilon} \cdot \vec{m} = 0$ et:
\begin{align}
S_i(\vec{x}_1) &= -\frac{m_i - \varepsilon_i}{\Gamma_t}+m_i \frac{e^{-t}}{\Gamma_t}\tanh\left( |\vec{m}|^2 \, \frac{e^{-t}}{\Gamma_t}\right) \\
&= -\frac{m_i - \varepsilon_i}{\Gamma_t}+m_i \frac{e^{-t}}{\Gamma_t}\tanh\left( \tilde{\mu}^2 d \, \frac{e^{-t}}{\Gamma_t}\right)
\end{align}

\subsection{Score au point $\vec{x}_2 = \vec{m} + \vec{\varepsilon}$}

\begin{align}
S_i(\vec{x}_2) &= -\frac{(m_i + \varepsilon_i)}{\Gamma_t}+m_i \frac{e^{-t}}{\Gamma_t}\tanh\left( (\vec{m} + \vec{\varepsilon}) \cdot \vec{m} \, \frac{e^{-t}}{\Gamma_t}\right) \\
&= -\frac{m_i + \varepsilon_i}{\Gamma_t}+m_i \frac{e^{-t}}{\Gamma_t}\tanh\left( (|\vec{m}|^2 + \vec{\varepsilon} \cdot \vec{m}) \, \frac{e^{-t}}{\Gamma_t}\right)
\end{align}

Si $\vec{\varepsilon}$ est orthogonal à $\vec{m}$:
\begin{align}
S_i(\vec{x}_2) &= -\frac{m_i + \varepsilon_i}{\Gamma_t}+m_i \frac{e^{-t}}{\Gamma_t}\tanh\left( \tilde{\mu}^2 d \, \frac{e^{-t}}{\Gamma_t}\right)
\end{align}

\subsection{Score au point $\vec{x}_3 = -\vec{m} - \vec{\varepsilon}$}

\begin{align}
S_i(\vec{x}_3) &= -\frac{(-m_i - \varepsilon_i)}{\Gamma_t}+m_i \frac{e^{-t}}{\Gamma_t}\tanh\left( (-\vec{m} - \vec{\varepsilon}) \cdot \vec{m} \, \frac{e^{-t}}{\Gamma_t}\right) \\
&= \frac{m_i + \varepsilon_i}{\Gamma_t}+m_i \frac{e^{-t}}{\Gamma_t}\tanh\left( (-|\vec{m}|^2 - \vec{\varepsilon} \cdot \vec{m}) \, \frac{e^{-t}}{\Gamma_t}\right)
\end{align}

Si $\vec{\varepsilon}$ est orthogonal à $\vec{m}$:
\begin{align}
S_i(\vec{x}_3) &= \frac{m_i + \varepsilon_i}{\Gamma_t}+m_i \frac{e^{-t}}{\Gamma_t}\tanh\left( -\tilde{\mu}^2 d \, \frac{e^{-t}}{\Gamma_t}\right) \\
&= \frac{m_i + \varepsilon_i}{\Gamma_t}-m_i \frac{e^{-t}}{\Gamma_t}\tanh\left( \tilde{\mu}^2 d \, \frac{e^{-t}}{\Gamma_t}\right)
\end{align}

\subsection{Score au point $\vec{x}_4 = -\vec{m} + \vec{\varepsilon}$}

\begin{align}
S_i(\vec{x}_4) &= -\frac{(-m_i + \varepsilon_i)}{\Gamma_t}+m_i \frac{e^{-t}}{\Gamma_t}\tanh\left( (-\vec{m} + \vec{\varepsilon}) \cdot \vec{m} \, \frac{e^{-t}}{\Gamma_t}\right) \\
&= \frac{m_i - \varepsilon_i}{\Gamma_t}+m_i \frac{e^{-t}}{\Gamma_t}\tanh\left( (-|\vec{m}|^2 + \vec{\varepsilon} \cdot \vec{m}) \, \frac{e^{-t}}{\Gamma_t}\right)
\end{align}

Si $\vec{\varepsilon}$ est orthogonal à $\vec{m}$:
\begin{align}
S_i(\vec{x}_4) &= \frac{m_i - \varepsilon_i}{\Gamma_t}+m_i \frac{e^{-t}}{\Gamma_t}\tanh\left( -\tilde{\mu}^2 d \, \frac{e^{-t}}{\Gamma_t}\right) \\
&= \frac{m_i - \varepsilon_i}{\Gamma_t}-m_i \frac{e^{-t}}{\Gamma_t}\tanh\left( \tilde{\mu}^2 d \, \frac{e^{-t}}{\Gamma_t}\right)
\end{align}

\section{Calcul du potentiel}

Le potentiel peut être calculé à partir du score par la relation:
\begin{align}
V(\vec{x}, t) = \frac{1}{2}\sum_i x_i^2 - 2\log P_t(\vec{x})
\end{align}

\subsection{Potentiel au point $\vec{x}_1 = \vec{m} - \vec{\varepsilon}$}

\begin{align}
V(\vec{x}_1, t) &= \frac{1}{2}|\vec{m} - \vec{\varepsilon}|^2 - 2\log P_t(\vec{x}_1) \\
&= \frac{1}{2}(|\vec{m}|^2 - 2\vec{m}\cdot\vec{\varepsilon} + |\vec{\varepsilon}|^2) - 2\log\left[\frac{1}{2\sqrt{2\pi\Gamma_t}^d}\left(e^{-A_1/(2\Gamma_t)} + e^{-B_1/(2\Gamma_t)}\right)\right] \\
&= \frac{1}{2}(|\vec{m}|^2 + |\vec{\varepsilon}|^2) - 2\log\left[\frac{1}{2\sqrt{2\pi\Gamma_t}^d}\left(e^{-A_1/(2\Gamma_t)} + e^{-B_1/(2\Gamma_t)}\right)\right]
\end{align}

si $\vec{\varepsilon}$ est orthogonal à $\vec{m}$.

\subsection{Potentiel au point $\vec{x}_2 = \vec{m} + \vec{\varepsilon}$}

\begin{align}
V(\vec{x}_2, t) &= \frac{1}{2}|\vec{m} + \vec{\varepsilon}|^2 - 2\log P_t(\vec{x}_2) \\
&= \frac{1}{2}(|\vec{m}|^2 + 2\vec{m}\cdot\vec{\varepsilon} + |\vec{\varepsilon}|^2) - 2\log\left[\frac{1}{2\sqrt{2\pi\Gamma_t}^d}\left(e^{-A_2/(2\Gamma_t)} + e^{-B_2/(2\Gamma_t)}\right)\right] \\
&= \frac{1}{2}(|\vec{m}|^2 + |\vec{\varepsilon}|^2) - 2\log\left[\frac{1}{2\sqrt{2\pi\Gamma_t}^d}\left(e^{-A_2/(2\Gamma_t)} + e^{-B_2/(2\Gamma_t)}\right)\right]
\end{align}

si $\vec{\varepsilon}$ est orthogonal à $\vec{m}$.

\subsection{Potentiel au point $\vec{x}_3 = -\vec{m} - \vec{\varepsilon}$}

\begin{align}
V(\vec{x}_3, t) &= \frac{1}{2}|-\vec{m} - \vec{\varepsilon}|^2 - 2\log P_t(\vec{x}_3) \\
&= \frac{1}{2}(|\vec{m}|^2 + 2\vec{m}\cdot\vec{\varepsilon} + |\vec{\varepsilon}|^2) - 2\log\left[\frac{1}{2\sqrt{2\pi\Gamma_t}^d}\left(e^{-A_3/(2\Gamma_t)} + e^{-B_3/(2\Gamma_t)}\right)\right] \\
&= \frac{1}{2}(|\vec{m}|^2 + |\vec{\varepsilon}|^2) - 2\log\left[\frac{1}{2\sqrt{2\pi\Gamma_t}^d}\left(e^{-A_3/(2\Gamma_t)} + e^{-B_3/(2\Gamma_t)}\right)\right]
\end{align}

si $\vec{\varepsilon}$ est orthogonal à $\vec{m}$.

\subsection{Potentiel au point $\vec{x}_4 = -\vec{m} + \vec{\varepsilon}$}

\begin{align}
V(\vec{x}_4, t) &= \frac{1}{2}|-\vec{m} + \vec{\varepsilon}|^2 - 2\log P_t(\vec{x}_4) \\
&= \frac{1}{2}(|\vec{m}|^2 - 2\vec{m}\cdot\vec{\varepsilon} + |\vec{\varepsilon}|^2) - 2\log\left[\frac{1}{2\sqrt{2\pi\Gamma_t}^d}\left(e^{-A_4/(2\Gamma_t)} + e^{-B_4/(2\Gamma_t)}\right)\right] \\
&= \frac{1}{2}(|\vec{m}|^2 + |\vec{\varepsilon}|^2) - 2\log\left[\frac{1}{2\sqrt{2\pi\Gamma_t}^d}\left(e^{-A_4/(2\Gamma_t)} + e^{-B_4/(2\Gamma_t)}\right)\right]
\end{align}

si $\vec{\varepsilon}$ est orthogonal à $\vec{m}$.

\section{Analyse du potentiel au temps critique}

Au temps critique de séparation $t_s \approx \frac{1}{2}\log d$, le potentiel développe une structure à double puits. Examinons cela en regardant le comportement du potentiel autour de chaque région.

Pour $t \gg t_s$, nous avons $\tanh\left(\tilde{\mu}^2 d \, \frac{e^{-t}}{\Gamma_t}\right) \approx \tilde{\mu}^2 d \, \frac{e^{-t}}{\Gamma_t}$ (car l'argument est petit), et le potentiel est essentiellement quadratique.

Pour $t \ll t_s$, nous avons $\tanh\left(\tilde{\mu}^2 d \, \frac{e^{-t}}{\Gamma_t}\right) \approx \text{sign}(\tilde{\mu}^2 d) = 1$ (car l'argument est grand), et le potentiel a une structure à deux puits.

La transition entre ces deux régimes se produit précisément lorsque $t \approx t_s$. À ce moment, les variances du potentiel autour des points $\pm \vec{m}$ deviennent négatives, conduisant à la bifurcation caractéristique de ce système.

\section{Conclusion}

Nous avons calculé les probabilités, les scores et les potentiels associés aux quatre points $\vec{m} \pm \vec{\varepsilon}$ et $-\vec{m} \pm \vec{\varepsilon}$. Cette analyse montre clairement la transition de phase qui caractérise le modèle de diffusion: au temps critique $t_s$, le potentiel passe d'une forme quadratique à une structure de double puits, conduisant à la séparation des classes.

Ce comportement illustre également pourquoi, en grande dimension, le temps critique de séparation $t_s$ est d'ordre $\frac{1}{2}\log d$, comme le montre l'analyse des valeurs propres de la matrice hessienne du potentiel.

\end{document}