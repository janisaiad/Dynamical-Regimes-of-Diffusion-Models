\documentclass[11pt,a4paper]{article}
\usepackage[utf8]{inputenc}
\usepackage[T1]{fontenc}
\usepackage{amsmath,amssymb,amsfonts}
\usepackage{graphicx}
\usepackage{xcolor}
\usepackage{bm}
\usepackage{physics}
\usepackage{hyperref}
\usepackage{subcaption}

\title{Calcul des Probabilités et Potentiels pour les Régimes Dynamiques des Modèles de Diffusion}
\author{Analyse Mathématique}
\date{\today}

\begin{document}

\maketitle

\section{Introduction}

Dans ce rapport, nous allons analyser le comportement des modèles de diffusion en calculant les probabilités et les potentiels associés à quatre points spécifiques. Cette analyse permettra de mieux comprendre les régimes dynamiques et les transitions critiques qui caractérisent ces modèles, notamment les temps de séparation ($t_s$) et de collapse ($t_c$).

\section{Définition du problème}

Considérons un modèle de mélange gaussien à deux composantes en dimension $d$, avec des centres situés à $\pm \vec{m}$ où $\vec{m} = \tilde{\mu}\sqrt{d}$. Nous souhaitons analyser le comportement de la distribution et du score en quatre points particuliers:
\begin{align}
\vec{x}_1 &= \vec{m} - \vec{\varepsilon} \\
\vec{x}_2 &= \vec{m} + \vec{\varepsilon} \\
\vec{x}_3 &= -\vec{m} - \vec{\varepsilon} \\
\vec{x}_4 &= -\vec{m} + \vec{\varepsilon}
\end{align}

où $\vec{\varepsilon}$ est un petit vecteur, typiquement orthogonal à $\vec{m}$.

\section{Calcul des probabilités}

Pour le modèle de mélange gaussien, la distribution à l'instant $t$ est donnée par:
\begin{align}
P_t(\vec{x}) = \frac{1}{2\sqrt{2\pi\Gamma_t}^d}\left[
e^{-(\vec{x}-\vec{m} e^{-t})^2/(2 \Gamma_t)}
+e^{-(\vec{x}+\vec{m} e^{-t})^2/(2 \Gamma_t)}
\right]
\end{align}

où $\Gamma_t=\sigma^2 e^{-2t} + \Delta_t$ et $\Delta_t = 1-e^{-2t}$.

\subsection{Probabilités aux quatre points}

Calculons maintenant $P_t(\vec{x})$ pour chacun des quatre points:

\subsubsection{Calcul pour $\vec{x}_1 = \vec{m} - \vec{\varepsilon}$}

\begin{align}
P_t(\vec{x}_1) &= \frac{1}{2\sqrt{2\pi\Gamma_t}^d}\left[
e^{-((\vec{m}-\vec{\varepsilon})-\vec{m} e^{-t})^2/(2 \Gamma_t)}
+e^{-((\vec{m}-\vec{\varepsilon})+\vec{m} e^{-t})^2/(2 \Gamma_t)}
\right] \\
&= \frac{1}{2\sqrt{2\pi\Gamma_t}^d}\left[
e^{-(\vec{m}(1-e^{-t})-\vec{\varepsilon})^2/(2 \Gamma_t)}
+e^{-(\vec{m}(1+e^{-t})-\vec{\varepsilon})^2/(2 \Gamma_t)}
\right]
\end{align}

Pour simplifier, notons:
\begin{align}
A_1 &= (\vec{m}(1-e^{-t})-\vec{\varepsilon})^2 \\
B_1 &= (\vec{m}(1+e^{-t})-\vec{\varepsilon})^2
\end{align}

\subsubsection{Calcul pour $\vec{x}_2 = \vec{m} + \vec{\varepsilon}$}

\begin{align}
P_t(\vec{x}_2) &= \frac{1}{2\sqrt{2\pi\Gamma_t}^d}\left[
e^{-((\vec{m}+\vec{\varepsilon})-\vec{m} e^{-t})^2/(2 \Gamma_t)}
+e^{-((\vec{m}+\vec{\varepsilon})+\vec{m} e^{-t})^2/(2 \Gamma_t)}
\right] \\
&= \frac{1}{2\sqrt{2\pi\Gamma_t}^d}\left[
e^{-(\vec{m}(1-e^{-t})+\vec{\varepsilon})^2/(2 \Gamma_t)}
+e^{-(\vec{m}(1+e^{-t})+\vec{\varepsilon})^2/(2 \Gamma_t)}
\right]
\end{align}

Nous notons:
\begin{align}
A_2 &= (\vec{m}(1-e^{-t})+\vec{\varepsilon})^2 \\
B_2 &= (\vec{m}(1+e^{-t})+\vec{\varepsilon})^2
\end{align}

\subsubsection{Calcul pour $\vec{x}_3 = -\vec{m} - \vec{\varepsilon}$}

\begin{align}
P_t(\vec{x}_3) &= \frac{1}{2\sqrt{2\pi\Gamma_t}^d}\left[
e^{-((-\vec{m}-\vec{\varepsilon})-\vec{m} e^{-t})^2/(2 \Gamma_t)}
+e^{-((-\vec{m}-\vec{\varepsilon})+\vec{m} e^{-t})^2/(2 \Gamma_t)}
\right] \\
&= \frac{1}{2\sqrt{2\pi\Gamma_t}^d}\left[
e^{-(-\vec{m}(1+e^{-t})-\vec{\varepsilon})^2/(2 \Gamma_t)}
+e^{-(-\vec{m}(1-e^{-t})-\vec{\varepsilon})^2/(2 \Gamma_t)}
\right]
\end{align}

Nous notons:
\begin{align}
A_3 &= (-\vec{m}(1+e^{-t})-\vec{\varepsilon})^2 \\
B_3 &= (-\vec{m}(1-e^{-t})-\vec{\varepsilon})^2
\end{align}

\subsubsection{Calcul pour $\vec{x}_4 = -\vec{m} + \vec{\varepsilon}$}

\begin{align}
P_t(\vec{x}_4) &= \frac{1}{2\sqrt{2\pi\Gamma_t}^d}\left[
e^{-((-\vec{m}+\vec{\varepsilon})-\vec{m} e^{-t})^2/(2 \Gamma_t)}
+e^{-((-\vec{m}+\vec{\varepsilon})+\vec{m} e^{-t})^2/(2 \Gamma_t)}
\right] \\
&= \frac{1}{2\sqrt{2\pi\Gamma_t}^d}\left[
e^{-(-\vec{m}(1+e^{-t})+\vec{\varepsilon})^2/(2 \Gamma_t)}
+e^{-(-\vec{m}(1-e^{-t})+\vec{\varepsilon})^2/(2 \Gamma_t)}
\right]
\end{align}

Nous notons:
\begin{align}
A_4 &= (-\vec{m}(1+e^{-t})+\vec{\varepsilon})^2 \\
B_4 &= (-\vec{m}(1-e^{-t})+\vec{\varepsilon})^2
\end{align}

\section{Calcul du score}

Le score du modèle de mélange gaussien est donné par:
\begin{align}
S_i(\vec{x}) = -\frac{x_i}{\Gamma_t}+m_i \frac{e^{-t}}{\Gamma_t}\tanh\left( \vec{x} \cdot \vec{m} \, \frac{e^{-t}}{\Gamma_t}\right)
\end{align}

Calculons maintenant le score pour chacun des quatre points.

\subsection{Score au point $\vec{x}_1 = \vec{m} - \vec{\varepsilon}$}

\begin{align}
S_i(\vec{x}_1) &= -\frac{(m_i - \varepsilon_i)}{\Gamma_t}+m_i \frac{e^{-t}}{\Gamma_t}\tanh\left( (\vec{m} - \vec{\varepsilon}) \cdot \vec{m} \, \frac{e^{-t}}{\Gamma_t}\right) \\
&= -\frac{m_i - \varepsilon_i}{\Gamma_t}+m_i \frac{e^{-t}}{\Gamma_t}\tanh\left( (|\vec{m}|^2 - \vec{\varepsilon} \cdot \vec{m}) \, \frac{e^{-t}}{\Gamma_t}\right)
\end{align}

Si $\vec{\varepsilon}$ est orthogonal à $\vec{m}$, alors $\vec{\varepsilon} \cdot \vec{m} = 0$ et:
\begin{align}
S_i(\vec{x}_1) &= -\frac{m_i - \varepsilon_i}{\Gamma_t}+m_i \frac{e^{-t}}{\Gamma_t}\tanh\left( |\vec{m}|^2 \, \frac{e^{-t}}{\Gamma_t}\right) \\
&= -\frac{m_i - \varepsilon_i}{\Gamma_t}+m_i \frac{e^{-t}}{\Gamma_t}\tanh\left( \tilde{\mu}^2 d \, \frac{e^{-t}}{\Gamma_t}\right)
\end{align}

\subsection{Score au point $\vec{x}_2 = \vec{m} + \vec{\varepsilon}$}

\begin{align}
S_i(\vec{x}_2) &= -\frac{(m_i + \varepsilon_i)}{\Gamma_t}+m_i \frac{e^{-t}}{\Gamma_t}\tanh\left( (\vec{m} + \vec{\varepsilon}) \cdot \vec{m} \, \frac{e^{-t}}{\Gamma_t}\right) \\
&= -\frac{m_i + \varepsilon_i}{\Gamma_t}+m_i \frac{e^{-t}}{\Gamma_t}\tanh\left( (|\vec{m}|^2 + \vec{\varepsilon} \cdot \vec{m}) \, \frac{e^{-t}}{\Gamma_t}\right)
\end{align}

Si $\vec{\varepsilon}$ est orthogonal à $\vec{m}$:
\begin{align}
S_i(\vec{x}_2) &= -\frac{m_i + \varepsilon_i}{\Gamma_t}+m_i \frac{e^{-t}}{\Gamma_t}\tanh\left( \tilde{\mu}^2 d \, \frac{e^{-t}}{\Gamma_t}\right)
\end{align}

\subsection{Score au point $\vec{x}_3 = -\vec{m} - \vec{\varepsilon}$}

\begin{align}
S_i(\vec{x}_3) &= -\frac{(-m_i - \varepsilon_i)}{\Gamma_t}+m_i \frac{e^{-t}}{\Gamma_t}\tanh\left( (-\vec{m} - \vec{\varepsilon}) \cdot \vec{m} \, \frac{e^{-t}}{\Gamma_t}\right) \\
&= \frac{m_i + \varepsilon_i}{\Gamma_t}+m_i \frac{e^{-t}}{\Gamma_t}\tanh\left( (-|\vec{m}|^2 - \vec{\varepsilon} \cdot \vec{m}) \, \frac{e^{-t}}{\Gamma_t}\right)
\end{align}

Si $\vec{\varepsilon}$ est orthogonal à $\vec{m}$:
\begin{align}
S_i(\vec{x}_3) &= \frac{m_i + \varepsilon_i}{\Gamma_t}+m_i \frac{e^{-t}}{\Gamma_t}\tanh\left( -\tilde{\mu}^2 d \, \frac{e^{-t}}{\Gamma_t}\right) \\
&= \frac{m_i + \varepsilon_i}{\Gamma_t}-m_i \frac{e^{-t}}{\Gamma_t}\tanh\left( \tilde{\mu}^2 d \, \frac{e^{-t}}{\Gamma_t}\right)
\end{align}

\subsection{Score au point $\vec{x}_4 = -\vec{m} + \vec{\varepsilon}$}

\begin{align}
S_i(\vec{x}_4) &= -\frac{(-m_i + \varepsilon_i)}{\Gamma_t}+m_i \frac{e^{-t}}{\Gamma_t}\tanh\left( (-\vec{m} + \vec{\varepsilon}) \cdot \vec{m} \, \frac{e^{-t}}{\Gamma_t}\right) \\
&= \frac{m_i - \varepsilon_i}{\Gamma_t}+m_i \frac{e^{-t}}{\Gamma_t}\tanh\left( (-|\vec{m}|^2 + \vec{\varepsilon} \cdot \vec{m}) \, \frac{e^{-t}}{\Gamma_t}\right)
\end{align}

Si $\vec{\varepsilon}$ est orthogonal à $\vec{m}$:
\begin{align}
S_i(\vec{x}_4) &= \frac{m_i - \varepsilon_i}{\Gamma_t}+m_i \frac{e^{-t}}{\Gamma_t}\tanh\left( -\tilde{\mu}^2 d \, \frac{e^{-t}}{\Gamma_t}\right) \\
&= \frac{m_i - \varepsilon_i}{\Gamma_t}-m_i \frac{e^{-t}}{\Gamma_t}\tanh\left( \tilde{\mu}^2 d \, \frac{e^{-t}}{\Gamma_t}\right)
\end{align}

\section{Calcul du potentiel}

Le potentiel peut être calculé à partir du score par la relation:
\begin{align}
V(\vec{x}, t) = \frac{1}{2}\sum_i x_i^2 - 2\log P_t(\vec{x})
\end{align}

\subsection{Potentiel au point $\vec{x}_1 = \vec{m} - \vec{\varepsilon}$}

\begin{align}
V(\vec{x}_1, t) &= \frac{1}{2}|\vec{m} - \vec{\varepsilon}|^2 - 2\log P_t(\vec{x}_1) \\
&= \frac{1}{2}(|\vec{m}|^2 - 2\vec{m}\cdot\vec{\varepsilon} + |\vec{\varepsilon}|^2) - 2\log\left[\frac{1}{2\sqrt{2\pi\Gamma_t}^d}\left(e^{-A_1/(2\Gamma_t)} + e^{-B_1/(2\Gamma_t)}\right)\right] \\
&= \frac{1}{2}(|\vec{m}|^2 + |\vec{\varepsilon}|^2) - 2\log\left[\frac{1}{2\sqrt{2\pi\Gamma_t}^d}\left(e^{-A_1/(2\Gamma_t)} + e^{-B_1/(2\Gamma_t)}\right)\right]
\end{align}

si $\vec{\varepsilon}$ est orthogonal à $\vec{m}$.

\subsection{Potentiel au point $\vec{x}_2 = \vec{m} + \vec{\varepsilon}$}

\begin{align}
V(\vec{x}_2, t) &= \frac{1}{2}|\vec{m} + \vec{\varepsilon}|^2 - 2\log P_t(\vec{x}_2) \\
&= \frac{1}{2}(|\vec{m}|^2 + 2\vec{m}\cdot\vec{\varepsilon} + |\vec{\varepsilon}|^2) - 2\log\left[\frac{1}{2\sqrt{2\pi\Gamma_t}^d}\left(e^{-A_2/(2\Gamma_t)} + e^{-B_2/(2\Gamma_t)}\right)\right] \\
&= \frac{1}{2}(|\vec{m}|^2 + |\vec{\varepsilon}|^2) - 2\log\left[\frac{1}{2\sqrt{2\pi\Gamma_t}^d}\left(e^{-A_2/(2\Gamma_t)} + e^{-B_2/(2\Gamma_t)}\right)\right]
\end{align}

si $\vec{\varepsilon}$ est orthogonal à $\vec{m}$.

\subsection{Potentiel au point $\vec{x}_3 = -\vec{m} - \vec{\varepsilon}$}

\begin{align}
V(\vec{x}_3, t) &= \frac{1}{2}|-\vec{m} - \vec{\varepsilon}|^2 - 2\log P_t(\vec{x}_3) \\
&= \frac{1}{2}(|\vec{m}|^2 + 2\vec{m}\cdot\vec{\varepsilon} + |\vec{\varepsilon}|^2) - 2\log\left[\frac{1}{2\sqrt{2\pi\Gamma_t}^d}\left(e^{-A_3/(2\Gamma_t)} + e^{-B_3/(2\Gamma_t)}\right)\right] \\
&= \frac{1}{2}(|\vec{m}|^2 + |\vec{\varepsilon}|^2) - 2\log\left[\frac{1}{2\sqrt{2\pi\Gamma_t}^d}\left(e^{-A_3/(2\Gamma_t)} + e^{-B_3/(2\Gamma_t)}\right)\right]
\end{align}

si $\vec{\varepsilon}$ est orthogonal à $\vec{m}$.

\subsection{Potentiel au point $\vec{x}_4 = -\vec{m} + \vec{\varepsilon}$}

\begin{align}
V(\vec{x}_4, t) &= \frac{1}{2}|-\vec{m} + \vec{\varepsilon}|^2 - 2\log P_t(\vec{x}_4) \\
&= \frac{1}{2}(|\vec{m}|^2 - 2\vec{m}\cdot\vec{\varepsilon} + |\vec{\varepsilon}|^2) - 2\log\left[\frac{1}{2\sqrt{2\pi\Gamma_t}^d}\left(e^{-A_4/(2\Gamma_t)} + e^{-B_4/(2\Gamma_t)}\right)\right] \\
&= \frac{1}{2}(|\vec{m}|^2 + |\vec{\varepsilon}|^2) - 2\log\left[\frac{1}{2\sqrt{2\pi\Gamma_t}^d}\left(e^{-A_4/(2\Gamma_t)} + e^{-B_4/(2\Gamma_t)}\right)\right]
\end{align}

si $\vec{\varepsilon}$ est orthogonal à $\vec{m}$.

\section{Analyse du potentiel au temps critique}

Au temps critique de séparation $t_s \approx \frac{1}{2}\log d$, le potentiel développe une structure à double puits. Examinons cela en regardant le comportement du potentiel autour de chaque région.

Pour $t \gg t_s$, nous avons $\tanh\left(\tilde{\mu}^2 d \, \frac{e^{-t}}{\Gamma_t}\right) \approx \tilde{\mu}^2 d \, \frac{e^{-t}}{\Gamma_t}$ (car l'argument est petit), et le potentiel est essentiellement quadratique.

Pour $t \ll t_s$, nous avons $\tanh\left(\tilde{\mu}^2 d \, \frac{e^{-t}}{\Gamma_t}\right) \approx \text{sign}(\tilde{\mu}^2 d) = 1$ (car l'argument est grand), et le potentiel a une structure à deux puits.

La transition entre ces deux régimes se produit précisément lorsque $t \approx t_s$. À ce moment, les variances du potentiel autour des points $\pm \vec{m}$ deviennent négatives, conduisant à la bifurcation caractéristique de ce système.

\section{Potentiel réduit et hiérarchie de séparation}

\subsection{Potentiel en fonction de la variable réduite $q$}

Pour mieux comprendre les transitions de phase dans le processus de diffusion, il est utile d'introduire la variable réduite $q = \frac{\vec{x} \cdot \vec{m}}{\sqrt{d}}$, qui représente le recouvrement normalisé entre le point $\vec{x}$ et la direction $\vec{m}$. Le potentiel peut alors s'exprimer comme une fonction de $q$ et $t$:

\begin{align}
V(q, t) = \frac{1}{2}q^2 - 2\tilde{\mu}^2 \log \cosh\left(q e^{-t} \sqrt{d}\right)
\end{align}

Cette formulation met en évidence que le potentiel dépend essentiellement de deux paramètres: le temps $t$ et la dimension $d$. La transition entre le régime à un seul puits et le régime à double puits est contrôlée par le produit $e^{-t}\sqrt{d}$.

\subsection{Temps critique de séparation}

Le temps critique de séparation $t_s$ correspond au moment où la courbure du potentiel à $q=0$ s'annule, c'est-à-dire $\frac{\partial^2 V}{\partial q^2}(q=0, t_s) = 0$. Cette condition nous donne:

\begin{align}
1 - 2\tilde{\mu}^2 d e^{-2t_s} = 0
\end{align}

ce qui conduit à:

\begin{align}
t_s = \frac{1}{2}\log(2\tilde{\mu}^2 d)
\end{align}

À des termes sous-dominants près, cela correspond à $t_s \approx \frac{1}{2}\log d$, comme mentionné précédemment.

\subsection{Effet de $\varepsilon$ et hiérarchie de séparation}

Lorsque l'on considère des perturbations $\vec{\varepsilon}$ qui ne sont pas orthogonales à $\vec{m}$, on obtient une hiérarchie de temps de séparation. Soit $\vec{\varepsilon} = \varepsilon_{\parallel}\frac{\vec{m}}{|\vec{m}|} + \vec{\varepsilon}_{\perp}$, où $\vec{\varepsilon}_{\perp}$ est orthogonal à $\vec{m}$. Le produit scalaire devient:

\begin{align}
\vec{x} \cdot \vec{m} &= (\pm\vec{m} \pm \vec{\varepsilon}) \cdot \vec{m}\\
&= \pm|\vec{m}|^2 \pm \varepsilon_{\parallel}|\vec{m}|\\
&= \pm\tilde{\mu}^2 d \pm \varepsilon_{\parallel}\tilde{\mu}\sqrt{d}
\end{align}

La variable réduite $q$ s'écrit alors:

\begin{align}
q &= \frac{\vec{x} \cdot \vec{m}}{\sqrt{d}}\\
&= \pm\tilde{\mu}^2\sqrt{d} \pm \varepsilon_{\parallel}\tilde{\mu}
\end{align}

Pour de grandes valeurs de $d$, le premier terme domine, mais la présence du terme additionnel $\varepsilon_{\parallel}\tilde{\mu}$ introduit une asymétrie dans le potentiel. Cette asymétrie crée une hiérarchie dans les barrières de potentiel, ce qui conduit à des temps de séparation différents pour différentes directions de $\vec{\varepsilon}$.

\subsection{Structure hiérarchique du potentiel}

L'asymétrie introduite par $\varepsilon_{\parallel}$ modifie la profondeur relative des puits de potentiel. Pour un vecteur $\vec{\varepsilon}$ avec une composante parallèle positive ($\varepsilon_{\parallel} > 0$), le puits autour de $+\vec{m}$ devient plus profond que celui autour de $-\vec{m}$, et vice versa pour $\varepsilon_{\parallel} < 0$.

Cette asymétrie conduit à une hiérarchie de temps critiques:

\begin{align}
t_s(\varepsilon_{\parallel}) \approx \frac{1}{2}\log\left(\frac{2\tilde{\mu}^2 d}{1 + 2\varepsilon_{\parallel}\tilde{\mu}/\sqrt{d}}\right)
\end{align}

Pour $\varepsilon_{\parallel} > 0$, le temps critique est légèrement inférieur à $t_s$, tandis que pour $\varepsilon_{\parallel} < 0$, il est légèrement supérieur. Cette différence, bien que petite pour $d$ grand, crée une séquence hiérarchique de transitions: d'abord la séparation entre les classes principales ($\pm\vec{m}$), puis des sous-séparations au sein de chaque classe.

\subsection{Régimes dynamiques après la séparation}

Après la transition de séparation, les trajectoires sont "engagées" vers l'un des centres $\pm\vec{m}$. Pour une trajectoire engagée vers $+\vec{m}$, le score se simplifie en:

\begin{align}
S_i^+(\vec{x}) = -\frac{x_i}{\Gamma_t} + m_i\frac{e^{-t}}{\Gamma_t}
\end{align}

C'est le score du processus backward d'une gaussienne simple centrée en $+\vec{m}$. Cela conduit à l'équation d'évolution:

\begin{align}
-dx_i = \left(-\frac{x_i}{\Gamma_t} + m_i\frac{e^{-t}}{\Gamma_t}\right)dt + d\eta_i(t)
\end{align}

où $d\eta_i(t)$ est la racine carrée de deux fois un mouvement brownien. Cette équation garantit que toutes les trajectoires évoluant selon elle généreront la gaussienne centrée en $+\vec{m}$.

De façon analogue, pour les trajectoires engagées vers $-\vec{m}$, on obtient la même équation avec $m_i$ remplacé par $-m_i$.

\section{Autosimilarité et structures hiérarchiques via SVD}

\subsection{SVD et axes principaux pour $\vec{\varepsilon}$ orthogonal}

Lorsque $\vec{\varepsilon}$ est orthogonal à $\vec{m}$, nous pouvons effectuer une décomposition en valeurs singulières (SVD) très élégante de la matrice de covariance du système. Considérons l'espace engendré par $\vec{m}$ et $\vec{\varepsilon}$. Ces deux vecteurs forment une base naturelle pour analyser le comportement du système.

La matrice de covariance $\Sigma$ du mélange gaussien peut s'écrire comme:

\begin{align}
\Sigma &= \mathbb{E}[(\vec{x}-\mathbb{E}[\vec{x}])(\vec{x}-\mathbb{E}[\vec{x}])^T] \\
&= \frac{1}{2}\left[ (\vec{m}-\vec{0})(\vec{m}-\vec{0})^T + (-\vec{m}-\vec{0})(-\vec{m}-\vec{0})^T \right] + \Gamma_t \mathbf{I} \\
&= \vec{m}\vec{m}^T + \Gamma_t \mathbf{I}
\end{align}

La SVD de cette matrice révèle deux directions principales:
\begin{enumerate}
    \item La direction de $\vec{m}$ avec valeur propre $\lambda_1 = |\vec{m}|^2 + \Gamma_t = \tilde{\mu}^2 d + \Gamma_t$
    \item Les directions orthogonales à $\vec{m}$ (dont $\vec{\varepsilon}$ fait partie) avec valeur propre $\lambda_2 = \Gamma_t$
\end{enumerate}

Cette structure des valeurs propres explique directement la hiérarchie observée: la séparation se produit d'abord dans la direction principale $\vec{m}$ (avec la plus grande variance), puis potentiellement dans les directions secondaires comme $\vec{\varepsilon}$.

\subsection{Scores conditionnels et potentiels autosimilaires}

Considérons maintenant les scores conditionnels, c'est-à-dire les scores calculés en supposant que le système s'est déjà engagé vers $+\vec{m}$ ou $-\vec{m}$. Pour une trajectoire qui collapse vers $+\vec{m}$, le score conditionnel est:

\begin{align}
S_i^{\text{cond}+}(\vec{x}) = -\frac{x_i - m_i e^{-t}}{\Gamma_t}
\end{align}

Ce score correspond à celui d'une distribution gaussienne centrée en $\vec{m}e^{-t}$ avec variance $\Gamma_t$. Le potentiel conditionnel correspondant est:

\begin{align}
V^{\text{cond}+}(\vec{x}, t) = \frac{1}{2}\sum_i \frac{(x_i - m_i e^{-t})^2}{\Gamma_t} + C
\end{align}

où $C$ est une constante.

Examinons maintenant ce qui se passe quand on applique une perturbation $\vec{\varepsilon}$ autour de $\vec{m}$. Si nous considérons le sous-espace orthogonal à $\vec{m}$ et centré en $\vec{m}$, nous pouvons définir une nouvelle variable:

\begin{align}
\vec{y} = \vec{x} - \vec{m}
\end{align}

Dans ce sous-espace, le potentiel conditionnel s'écrit:

\begin{align}
V^{\text{cond}+}(\vec{y} + \vec{m}, t) = \frac{1}{2}\sum_i \frac{(y_i + m_i(1-e^{-t}))^2}{\Gamma_t} + C
\end{align}

Pour $t \gg 0$, le terme $m_i(1-e^{-t}) \approx m_i$, et le potentiel devient:

\begin{align}
V^{\text{cond}+}(\vec{y} + \vec{m}, t) \approx \frac{1}{2}\sum_i \frac{(y_i + m_i)^2}{\Gamma_t} + C
\end{align}

Avec $\vec{y}$ orthogonal à $\vec{m}$ (comme $\vec{\varepsilon}$), nous avons $\sum_i y_i m_i = 0$, et le potentiel se simplifie:

\begin{align}
V^{\text{cond}+}(\vec{y} + \vec{m}, t) \approx \frac{1}{2}\sum_i \left(\frac{y_i^2}{\Gamma_t} + \frac{m_i^2}{\Gamma_t}\right) + C
\end{align}

Cette forme est remarquable: le premier terme est un potentiel quadratique en $\vec{y}$, tout comme le potentiel original en $\vec{x}$ pour $t \gg t_s$ (avant la séparation).

\subsection{Autosimilarité et fractales dans les modèles de diffusion}

L'observation précédente révèle une propriété fascinante d'autosimilarité: après conditionnement sur la séparation principale (selon $\vec{m}$), le potentiel dans le sous-espace orthogonal présente la même structure que le potentiel original à des temps antérieurs.

Si nous introduisons une perturbation $\vec{\varepsilon}$ avec une composante parallèle à elle-même, par exemple $\vec{\alpha} = \alpha \frac{\vec{\varepsilon}}{|\vec{\varepsilon}|}$, nous obtenons un deuxième niveau de séparation, similaire au premier niveau mais à une échelle différente.

Cette structure se répète potentiellement à plusieurs niveaux, créant un comportement de type fractal dans l'espace des configurations. Pour que cette autosimilarité soit exacte, le rapport entre les échelles successives doit être constant. Cela nous conduit à un critère sur $\vec{\varepsilon}$:

\begin{align}
\frac{|\vec{\varepsilon}|^2}{\Gamma_t} = \frac{|\vec{m}|^2}{\Gamma_{t_0}}
\end{align}

où $t_0$ est le temps initial (avant la première séparation) et $t$ le temps après la première séparation. Avec $|\vec{m}|^2 = \tilde{\mu}^2 d$ et $\Gamma_t \approx 1$ pour $t$ grand, cela donne:

\begin{align}
|\vec{\varepsilon}|^2 = \frac{\Gamma_t}{\Gamma_{t_0}} \tilde{\mu}^2 d \approx \frac{\tilde{\mu}^2 d}{\Gamma_{t_0}}
\end{align}

Pour $t_0 = 0$, nous avons $\Gamma_{t_0} = \sigma^2$ (la variance initiale), ce qui donne:

\begin{align}
|\vec{\varepsilon}| = \tilde{\mu}\sqrt{\frac{d}{\sigma^2}}
\end{align}

Ce critère garantit que le modèle présente une autosimilarité parfaite, avec des séparations qui se produisent à des échelles successivement réduites, mais avec la même structure dynamique.

\subsection{Vérifions numériquement}

Pour vérifier cette autosimilarité, considérons le potentiel conditionnel pour un vecteur $\vec{x} = \vec{m} + q\vec{\varepsilon}$ où $q$ varie. Le potentiel conditionnel s'écrit:

\begin{align}
V^{\text{cond}+}(\vec{m} + q\vec{\varepsilon}, t) = \frac{1}{2}\frac{q^2|\vec{\varepsilon}|^2}{\Gamma_t} + C
\end{align}

Si nous ajoutons maintenant une perturbation $\vec{\alpha} = \alpha \frac{\vec{\varepsilon}}{|\vec{\varepsilon}|}$ avec $\alpha > 0$ et $\alpha < 0$, nous obtenons un potentiel de mélange gaussien à deux composantes:

\begin{align}
V^{\text{cond}2}(q) = -\log\left[e^{-\frac{(q-\alpha)^2|\vec{\varepsilon}|^2}{2\Gamma_t}} + e^{-\frac{(q+\alpha)^2|\vec{\varepsilon}|^2}{2\Gamma_t}}\right] + C'
\end{align}

Ce potentiel présente une structure à double puits lorsque:

\begin{align}
\frac{\alpha^2|\vec{\varepsilon}|^2}{\Gamma_t} > 1
\end{align}

Ce qui conduit à un nouveau temps critique $t_{s2}$ pour cette séparation secondaire:

\begin{align}
t_{s2} = \frac{1}{2}\log\left(\alpha^2|\vec{\varepsilon}|^2\right)
\end{align}

Si nous choisissons $|\vec{\varepsilon}| = \tilde{\mu}\sqrt{\frac{d}{\sigma^2}}$ et $\alpha = \sigma$, nous obtenons:

\begin{align}
t_{s2} = \frac{1}{2}\log\left(\sigma^2 \cdot \tilde{\mu}^2\frac{d}{\sigma^2}\right) = \frac{1}{2}\log\left(\tilde{\mu}^2 d\right) = t_s
\end{align}

Ainsi, le temps critique de la séparation secondaire est identique au temps critique de la séparation primaire, confirmant l'autosimilarité parfaite du système!

\begin{figure}[h]
    \centering
    \caption{Comparaison des potentiels à différentes échelles montrant l'autosimilarité}
    \label{fig:self_similarity}
\end{figure}

\subsection{Matrice jacobienne et structures hiérarchiques}

Une autre manière d'analyser cette autosimilarité est d'examiner la matrice jacobienne du score $\nabla S(\vec{x})$. Pour le mélange gaussien, cette matrice est:

\begin{align}
\nabla S(\vec{x}) = -\frac{1}{\Gamma_t}\mathbf{I} + \frac{e^{-2t}}{\Gamma_t^2}\vec{m}\vec{m}^T\left(1 - \tanh^2\left(\vec{x}\cdot\vec{m}\frac{e^{-t}}{\Gamma_t}\right)\right)
\end{align}

Cette matrice a des valeurs propres différentes selon les directions. Au temps critique $t_s$, la valeur propre dans la direction $\vec{m}$ s'annule, créant l'instabilité qui conduit à la séparation. Après conditionnement sur cette séparation, la jacobienne du score conditionnel dans le sous-espace orthogonal devient:

\begin{align}
\nabla S^{\text{cond}}(\vec{y}) = -\frac{1}{\Gamma_t}\mathbf{I}
\end{align}

Cette forme est identique à celle du score original à $t \gg t_s$, confirmant encore l'autosimilarité du système.

\section{Analyse numérique du phénomène de double puits}

Pour visualiser la formation du double puits et la hiérarchie des temps de séparation, nous pouvons étudier numériquement le potentiel $V(q,t)$ pour différentes valeurs de $t$ et différentes perturbations $\varepsilon$.

\subsection{Visualisation du potentiel pour différentes valeurs de $t$}

La figure \ref{fig:potential_evolution} illustre l'évolution du potentiel $V(q,t)$ en fonction du temps normalisé $t/t_s$. On observe clairement la transition d'un potentiel quadratique (pour $t \gg t_s$) à un potentiel à double puits (pour $t \ll t_s$).

\begin{figure}[h]
    \centering
    \caption{Évolution du potentiel $V(q,t)$ en fonction du temps $t/t_s$}
    \label{fig:potential_evolution}
\end{figure}

\subsection{Effet de la perturbation $\varepsilon$ sur le potentiel}

Lorsque $\vec{\varepsilon}$ a une composante non nulle dans la direction de $\vec{m}$, l'asymétrie du potentiel devient visible. La figure \ref{fig:potential_asymmetry} montre le potentiel pour différentes valeurs de $\varepsilon_{\parallel}$.

\begin{figure}[h]
    \centering
    \begin{subfigure}{0.45\textwidth}
        \centering
        \caption{$\varepsilon_{\parallel} = 0$}
        \label{fig:eps_0}
    \end{subfigure}
    \begin{subfigure}{0.45\textwidth}
        \centering
        \caption{$\varepsilon_{\parallel} = 0.1\tilde{\mu}\sqrt{d}$}
        \label{fig:eps_pos}
    \end{subfigure}
    
    \begin{subfigure}{0.45\textwidth}
        \centering
        \caption{$\varepsilon_{\parallel} = -0.1\tilde{\mu}\sqrt{d}$}
        \label{fig:eps_neg}
    \end{subfigure}
    \caption{Asymétrie du potentiel pour différentes valeurs de $\varepsilon_{\parallel}$ à $t = 0.8 t_s$}
    \label{fig:potential_asymmetry}
\end{figure}

\subsection{Hiérarchie des temps critiques}

En observant l'évolution de la courbure du potentiel autour des points $\pm \vec{m} \pm \vec{\varepsilon}$, on peut déterminer les temps critiques associés à chacun de ces points. La table \ref{tab:critical_times} présente ces temps pour différentes valeurs de $\varepsilon_{\parallel}$.

\begin{table}[h]
    \centering
    \begin{tabular}{|c|c|}
        \hline
        Point & Temps critique \\
        \hline
        $\vec{m} + \vec{\varepsilon}$ ($\varepsilon_{\parallel} > 0$) & $t_s - \Delta t^+$ \\
        $\vec{m} - \vec{\varepsilon}$ ($\varepsilon_{\parallel} < 0$) & $t_s + \Delta t^-$ \\
        $-\vec{m} + \vec{\varepsilon}$ ($\varepsilon_{\parallel} > 0$) & $t_s + \Delta t^+$ \\
        $-\vec{m} - \vec{\varepsilon}$ ($\varepsilon_{\parallel} < 0$) & $t_s - \Delta t^-$ \\
        \hline
    \end{tabular}
    \caption{Temps critiques pour les quatre points spéciaux}
    \label{tab:critical_times}
\end{table}

où $\Delta t^{\pm} \propto \varepsilon_{\parallel}/\sqrt{d}$ représente le décalage temporel dû à l'asymétrie.

\section{Conclusion}

Nous avons calculé les probabilités, les scores et les potentiels associés aux quatre points $\vec{m} \pm \vec{\varepsilon}$ et $-\vec{m} \pm \vec{\varepsilon}$. Cette analyse montre clairement la transition de phase qui caractérise le modèle de diffusion: au temps critique $t_s$, le potentiel passe d'une forme quadratique à une structure de double puits, conduisant à la séparation des classes.

En introduisant la variable réduite $q = \vec{x} \cdot \vec{m} / \sqrt{d}$, nous avons obtenu une expression compacte du potentiel qui met en évidence son comportement en fonction du temps et de la dimension. L'analyse de l'effet des perturbations non orthogonales à $\vec{m}$ révèle une structure hiérarchique des temps critiques, créant une séquence de transitions au sein de chaque classe principale.

Ce comportement hiérarchique illustre pourquoi, même en grande dimension, la dynamique du modèle de diffusion peut présenter des sous-structures complexes. La compréhension de ces phénomènes est essentielle pour optimiser l'utilisation des modèles de diffusion dans des applications pratiques.

\end{document}