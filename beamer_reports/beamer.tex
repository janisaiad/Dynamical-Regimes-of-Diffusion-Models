% !TEX TS-program = pdflatex
% !TEX encoding = UTF-8

\documentclass[aspectratio=169]{beamer}
\usetheme{Madrid}
\usecolortheme{whale}
\usepackage[utf8]{inputenc}
\usepackage[T1]{fontenc}
\usepackage{amsmath,amssymb,amsfonts}
\usepackage{graphicx}
\usepackage{xcolor}
\usepackage{bm}
\usepackage{tikz}
\usepackage{physics}
\usepackage{algorithm}
\usepackage{algorithmic}
\usepackage{graphicx}
% Color definitions
\definecolor{myblue}{RGB}{0,114,178}
\definecolor{myred}{RGB}{213,94,0}
\definecolor{mygreen}{RGB}{0,158,115}

\title{Dynamic Regimes of Diffusion Models}
\subtitle{A Volumetric and Temporal Analysis}
\author{Janis Aiad \and Thomas Pouponneau \and Aurélien Arnoux}
\institute{EA Topics in ML \\ Ecole Polytechnique \\ \vspace{0.5cm} \small Based on: Biroli et al., Nature Communications 15.1 (2024)}
\date{\today}

\begin{document}

\begin{frame}
    \titlepage
    \vspace{-1cm}
    \begin{center}
        \footnotesize
        \begin{tabular}{p{0.9\textwidth}}
            \textbf{Reference:} \\
            @article\{Biroli\_2024, \\
            \hspace{0.3cm} title=\{Dynamical regimes of diffusion models\}, \\
            \hspace{0.3cm} journal=\{Nature Communications\}, \\
            \hspace{0.3cm} author=\{Biroli, Giulio and Bonnaire, Tony and de Bortoli, Valentin and Mézard, Marc\}, \\
            \hspace{0.3cm} year=\{2024\}, volume=\{15\}, number=\{1\} \\
            \}
        \end{tabular}
    \end{center}
\end{frame}

\begin{frame}{Outline}
    \tableofcontents
\end{frame}



\section{Introduction}


%% 30 s
\begin{frame}{Example}
       
    \begin{center}
        \fbox{\parbox{0.8\textwidth}{
            \centering
            \vspace{2cm}
            [Image: car diffusion animation ]
            \vspace{2cm}
        }}
    \end{center}
\end{frame}

%% 1 minute
\begin{frame}{Diffusion Models: Mathematical Foundations}
    \begin{itemize}
        \item Dataset of images $x_i \in \mathbb{R}^d$ with d large (e.g. 784 for MNIST)
         \begin{equation}
            d \gg 1 \text{ (typically } 10^3-10^4\text{)}
        \end{equation}
        \item Stochastic Differential Equation (SDE):
        \begin{equation}
            dx_t = -\frac{1}{2} x_t dt + \sqrt{2} dW_t
        \end{equation}
        
        
        \item Two intial and forwarded probability distributions:
        \begin{itemize}
            \item Empirical distribution with $n$ samples (data): $p_0^e(x) = \frac{1}{n} \sum_{i=1}^n \delta(x-x_i)$ and $p_t^e(x) = \frac{1}{n} \sum_{i=1}^n \delta(x-x_i(t))$
            \item Underlying distribution to learn (manifold): $p_0(x) = \frac{1}{Z_0} e^{-\frac{x^2}{2\sigma_0^2}}$ and $p_t(x) = \frac{1}{Z_t} e^{-\frac{x^2}{2\sigma_t^2}}$
        \end{itemize}
        
    \end{itemize}
\end{frame}

%% 1 minute
\begin{frame}{Motivation}
    \begin{itemize}
        
        \item Gaussian convolution transformation:
        \begin{equation}
            p_t(x) = \int dy \; p_0(y) \frac{e^{-(x-ye^{-t})^2/(2\sigma_t^2)}}{(2\pi\sigma_t^2)^{d/2}}
        \end{equation}
        where $\sigma_t^2 = 1-e^{-2t}$

        \item Exact score learned hypothesis:
        \begin{equation}
            \nabla \log p_t(x) \approx \hat{\nabla} \log p_t^e(x)
        \end{equation}

        \item Key questions :
        \begin{itemize}
            \item Trends of $p_t(x)$ and $p_t^e(x)$ with t and d
            \item How to determine optimal beginning/stopping times?
        \end{itemize}
    \end{itemize}
\end{frame}



%% 30 s
\begin{frame}{Goal}
    \begin{itemize}
        \item Two main critical times:
        \begin{itemize}
            \item $t_s$: Critical time for class separation (gender, lighting ...)
            \item 2 proofs, 1 with use of score along eigenvectors, 1 with use of covariance and linear separability
        \end{itemize}
        
        \begin{center}
            \fbox{\parbox{0.8\textwidth}{
                \centering
                \vspace{2cm}
                [Image: t_c animation with real images]
                \vspace{2cm}
            }}
        \end{center}
        
        \begin{itemize}
            \item $t_c$: Critical time for image selection
        \end{itemize}
           
        \begin{center}
            \fbox{\parbox{0.8\textwidth}{
                \centering
                \vspace{2cm}
                [Image: Better viz from the paper in tikz, with true generated data]
                \vspace{2cm}
            }}
        \end{center}
    \end{itemize}
\end{frame}











\section{Speciation time : crafting your characteristics}

%% 2 minute
\begin{frame}{Critical Separation Time ($t_s$)}
    \begin{itemize}
    
        \item 2 covariance matrices:
        
        \item Covariance C $\approx$ Cov(samples) = diag$(\lambda_1, \lambda_2, \lambda_3, \lambda_4, \epsilon, ..., \epsilon)$ with $\lambda_i \gg \epsilon$ = clusters infos
        
        \item Represents when linear separability of classes begin to appear (like embedding space)
        
        \item forwarding, we have 
        \begin{equation}
            Cov(p_t^e(x)) = E_{x_i \sim p_t^e(x)} [(x_i - \bar{x})(x_i - \bar{x})^T]
        \end{equation}
        
        \item With data, can be estimated with :
        \begin{equation}
            Cov(p_t^e(x)) = \frac{1}{n} \sum_{i=1}^n (x_i - \bar{x})(x_i - \bar{x})^T
        \end{equation}
        
        \item In general, trend:
        \begin{equation}
            t_s \approx \frac{1}{2}\log(d)
        \end{equation}
    \end{itemize}
\end{frame}
    

\begin{frame}{Example curves in 1D/2D}
    
    \item Wall potentials - score along eigenvectors
    \item In 1D/2D: clear observation of 2 critical times
    \begin{center}
        \fbox{\parbox{0.8\textwidth}{
            \centering
            \vspace{2cm}
            [Image: Class separation visualization]
            \vspace{2cm}
        }}
    \end{center}
\end{frame}



%% 2 minute
\begin{frame}{Results on Synthetic Data}
    \begin{itemize}
        \item In higher dimensions: behavior confirms theoretical predictions
    \end{itemize}
    
    \begin{center}
        \fbox{\parbox{0.8\textwidth}{
            \centering
            \vspace{2cm}
            [Image: MNIST class separation visualization]
            \vspace{2cm}
        }}
    \end{center}
\end{frame}


%% 2 minute
\begin{frame}{Results on MNIST, materials}
    \begin{itemize}
        \item Spectral estimate of $t_s$ consistent with the dataset
        \item Potential wall clearly visible
    \end{itemize}

    
    
    \begin{center}
        \fbox{\parbox{0.8\textwidth}{
            \centering
            \vspace{2cm}
            [Image: MNIST Linear accuracy as a function of time + theoretical predictions (with integral of erfc(x) + empirical potential given centroids ]
            \vspace{2cm}
        }}
    \end{center}
\end{frame}















\section{Collapse time : avoid memorization}

\begin{frame}{Example gif}
    \begin{itemize}
    \end{itemize}
    \begin{center}
        \fbox{\parbox{0.8\textwidth}{
            \centering
            \vspace{2cm}
            [Image: CIFAR 2 images separation visualization]
            \vspace{2cm}
        }}
    \end{center}
    \begin{itemize}
        \item $t \leq t_c :$ distribution concentrated to training points = regurgitation
        \item $t > t_c :$ distribution concentrated to circle around training points
    \end{itemize}
\end{frame}


%% 1 minute
\begin{frame}{Critical Collapse Time ($t_c$)}
    \begin{itemize}
        \item Idea : stop generation before $t_c$
        \item Goal : get a "barycenter" between 2 near images
        \item $p_t^e(x) \propto Z_1 + Z_{2:n}$ where:
        \begin{itemize}
            \item $Z_1 \simeq e^{-d/2}$ (large $d$ limit)
            \item $Z_{2:n}$ concentrates on $e^{d\psi_+(t)}$ with $\psi_+(t)$ increasing in $t$ and $\alpha$
        \end{itemize}
        \item When $\alpha = \frac{\log n}{d}$ is small:
        \begin{equation}
            t_c \approx \frac{1}{2}\log\left[1+\frac{\sigma^2}{n^{2/d}-1}\right] \approx \frac{d}{\log n}
        \end{equation}
    \end{itemize}
    
    \begin{center}
        \fbox{\parbox{0.8\textwidth}{
            \centering
            \vspace{2cm}
            [Image: Collapse visualization of stopping between 2 images with CIFAR]
            \vspace{2cm}
        }}
    \end{center}
\end{frame}



%% 2 minute
\begin{frame}{Critical Collapse Time ($t_c$)}
\begin{itemize}
    \item Intuition : more data = more choice to memorize an image !
\end{itemize}
\begin{center}
        \fbox{\parbox{0.8\textwidth}{
            \centering
            \vspace{2cm}
            [Image: Collapse visualization of stopping when near but not exactly on a 2D point]
            \vspace{2cm}
        }}
    \end{center}
\end{frame}


%% 1 minute
\begin{frame}{The Volumetric Argument}
    \begin{itemize}
        \item Main idea - use the curse of dimensionality - mass concentrated not around a point to avoid collapsing
    
        \item Allow analysis without exact score depends on:
        \begin{itemize}
            \item $n$: number of samples
            \item $d$: data dimension
            \item Capacity of the model used to learn the score
        \end{itemize}
    \end{itemize}
\end{frame}

%% 1 minute
\begin{frame}{The Volumetric Argument}
    
    \begin{center}
        \fbox{\parbox{0.8\textwidth}{
            \centering
            \vspace{2cm}
            [Image: MNIST or CIFAR 2 images separation visualization]
            \vspace{2cm}
        }}
    \end{center}
\end{frame}










\section{Conclusion and what to remember}

%% 1 minute
\begin{frame}{Summary}
    \begin{itemize}
        \item Theory : You can calculate $t_c$ and $t_s$ for any new dataset, get a good time scheme
        \item Practice : Faster generation after training (HuggingFace)
    \end{itemize}

    \begin{itemize}
        \item Volumetric argument applicable beyond the exact empirical score hypothesis
        \item Deep understanding of model behavior as a function of $n$ and $d$
    \end{itemize}
\end{frame}


%% 1 minute
\begin{frame}{Future Directions - Very trendy subject}
    \begin{itemize}
        \item Data : hierarchy - distinguishing collapse/speciation
        \item Autosimilar Timescale for score, 4 walls potential
        \item Flow matching - same phenomena (cusps)
        \item Inexact score  = reduce collapse
        \item very recent results (1 months) on manifolds ($t_c = \frac{d_{manifold}}{\log(n)}$)
    \end{itemize}
    
    \begin{center}
        \fbox{\parbox{0.8\textwidth}{
            \centering
            \vspace{2cm}
            [Image: Hierarchical visualization with hierarchical tikz + calculations]
            \vspace{2cm}
        }}
    \end{center}
\end{frame}



























\begin{frame}
    \centering
    \LARGE Thank you for your attention!
    
    \vspace{1cm}
    
    \large Questions?

    
    \begin{center}
        \fbox{\parbox{0.8\textwidth}{
            \centering
            \vspace{2cm}
            [Image: Better viz from the paper in tikz, with true generated data]
            \vspace{2cm}
        }}
    \end{center}

        
\end{frame}

\end{document} 














