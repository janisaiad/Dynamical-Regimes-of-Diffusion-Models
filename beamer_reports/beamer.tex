% !TEX TS-program = pdflatex
% !TEX encoding = UTF-8

\documentclass[aspectratio=169]{beamer}
\usetheme{Madrid}
\usecolortheme{whale}
\usepackage[utf8]{inputenc}
\usepackage[T1]{fontenc}
\usepackage{amsmath,amssymb,amsfonts}
\usepackage{graphicx}
\usepackage{xcolor}
\usepackage{bm}
\usepackage{tikz}
\usepackage{physics}
\usepackage{algorithm}
\usepackage{algorithmic}

% Color definitions
\definecolor{myblue}{RGB}{0,114,178}
\definecolor{myred}{RGB}{213,94,0}
\definecolor{mygreen}{RGB}{0,158,115}

\title{Dynamic Regimes of Diffusion Models}
\subtitle{A Volumetric and Temporal Analysis}
\author{Scientific Presentation}
\institute{Research Laboratory}
\date{\today}

\begin{document}

\begin{frame}
    \titlepage
\end{frame}

\begin{frame}{Outline}
    \tableofcontents
\end{frame}

\section{Introduction}

\begin{frame}{Context and Motivation}
    \begin{itemize}
        \item Diffusion models have become essential in image generation
        \item Understanding their dynamics is crucial for optimizing their performance
        \item Key questions:
        \begin{itemize}
            \item How do distributions evolve during the diffusion process?
            \item What are the critical regimes that define their behavior?
            \item How to determine optimal stopping times?
        \end{itemize}
    \end{itemize}
\end{frame}

\begin{frame}{Theoretical Foundations}
    \begin{itemize}
        \item Diffusion equation:
        \begin{equation}
            dx_t = -\frac{1}{2}\beta_t x_t dt + \sqrt{\beta_t} dW_t
        \end{equation}
        
        \item Empirical score hypothesis:
        \begin{equation}
            \nabla \log p_t(x) \approx \hat{\nabla} \log p_t(x)
        \end{equation}
        
        \item Two main critical times:
        \begin{itemize}
            \item $t_c$ : critical time of volumetric collapse
            \item $t_s$ : critical time of class separation
        \end{itemize}
    \end{itemize}
\end{frame}

\section{Volumetric Analysis}

\begin{frame}{The Volumetric Argument}
    \begin{itemize}
        \item Beyond the "exact empirical score" hypothesis
        \item Allows quantitative analysis of how collapse depends on:
        \begin{itemize}
            \item $n$ : number of samples
            \item $d$ : data dimension
            \item Capacity of the model used to learn the score
        \end{itemize}
        \item Expression of critical collapse time:
        \begin{equation}
            t_c \sim \log\left(\frac{n}{d}\right)
        \end{equation}
    \end{itemize}
\end{frame}

\begin{frame}{Dynamics on Manifolds}
    \begin{itemize}
        \item Real data often lies on low-dimensional manifolds
        \item For a manifold of intrinsic dimension $d_{int} \ll d$:
        \begin{equation}
            t_c \sim \log\left(\frac{n}{d_{int}}\right)
        \end{equation}
        \item Important implications for learning and generation
        \item Allows defining a stopping strategy even if $t_c$ is large
    \end{itemize}
\end{frame}

\section{Critical Times}

\begin{frame}{Critical Collapse Time ($t_c$)}
    \begin{itemize}
        \item Calculated using the volumetric argument
        \item Represents when the distribution becomes too concentrated
        \item Strongly depends on the samples/dimension ratio
        \item Empirical formula:
        \begin{equation}
            t_c \approx \log\left(\frac{n \cdot \sigma^2}{d}\right)
        \end{equation}
        where $\sigma^2$ is the data variance
    \end{itemize}
\end{frame}

\begin{frame}{Critical Separation Time ($t_s$)}
    \begin{itemize}
        \item Calculated using the empirical covariance matrix
        \item Represents when classes begin to mix
        \item For multimodal distributions (e.g., multiple classes)
        \item Possible hierarchical structure (MNIST: different $t_s$ between digits)
        \item Can be estimated by:
        \begin{equation}
            t_s \approx \log\left(\frac{\Delta\mu^2}{\sigma^2}\right)
        \end{equation}
        where $\Delta\mu$ is the distance between class means
    \end{itemize}
\end{frame}

\section{Experiments and Results}

\begin{frame}{Experimental Methodology}
    \begin{itemize}
        \item Analysis of synthetic manifolds
        \item Tests on MNIST to calculate $t_c$ and $t_s$
        \item Study of dimension $d$ effect
        \item Exploration of multidimensional potentials
        \item Comparisons with flow matching
        \item Visualization of diffusion regimes
    \end{itemize}
\end{frame}

\begin{frame}{Results on Synthetic Data}
    \begin{itemize}
        \item Multi-well potentials
        \item In 1D: clear observation of two critical times
        \item In higher dimensions: behavior consistent with theoretical predictions
        \item Confirmation of the relation $t_c \sim \log(n/d)$
        \item For synthetic manifolds: critical time depends on intrinsic dimension
    \end{itemize}
\end{frame}

\begin{frame}{Results on MNIST}
    \begin{itemize}
        \item Intrinsic dimension estimated by non-parametric methods (UMAP, t-SNE)
        \item Hierarchical structure of $t_s$ between different digits
        \item Some digit pairs mix faster (e.g., 1-7)
        \item Calculation of $t_c$ consistent with volumetric formula
        \item Possibility to stop the process before $t_c$ due to low intrinsic dimension
    \end{itemize}
\end{frame}

\section{Future Developments}

\begin{frame}{Beyond Exact Empirical Score}
    \begin{itemize}
        \item The volumetric argument works even if the score isn't exact
        \item Several approaches to improve generation:
        \begin{itemize}
            \item Deconvolution with image noise addition
            \item Manifold interpolation
            \item Ensemble training to mitigate negative effects
            \item Score regularization
            \item Use of high learning rates
        \end{itemize}
    \end{itemize}
\end{frame}

\begin{frame}{Regularization and Improvement}
    \begin{itemize}
        \item A poorly learned but regularized score can reduce collapse
        \item Quantitative study of regularization role
        \item Flow matching: verify if it presents the same phenomena (cusps)
        \item Non-parametric identification of critical zones during training
        \item Tests on multi-layer synthetic manifolds
    \end{itemize}
\end{frame}

\begin{frame}{Practical Applications}
    \begin{itemize}
        \item Calculation of $t_c$ and $t_s$ for any new dataset
        \item Use of pre-trained models (HuggingFace)
        \item Consistency verification on various datasets
        \item Hierarchical speciation between classes
        \item Optimization of stopping times for generation
    \end{itemize}
\end{frame}

\section{Conclusion}

\begin{frame}{Summary of Contributions}
    \begin{itemize}
        \item Characterization of dynamic regimes in diffusion models
        \item Identification and quantification of critical times $t_c$ and $t_s$
        \item Validation on synthetic and real data (MNIST)
        \item Volumetric argument applicable beyond empirical score hypothesis
        \item Deep understanding of model behavior based on $n$ and $d$
    \end{itemize}
\end{frame}

\begin{frame}{Perspectives}
    \begin{itemize}
        \item In-depth study of regularization methods
        \item Generalization to other generative model architectures
        \item Applications beyond computer vision
        \item Optimization of stopping strategies for different data types
        \item Open-source implementation of critical time calculation methods
    \end{itemize}
\end{frame}

\begin{frame}
    \centering
    \LARGE Thank you for your attention!
    
    \vspace{1cm}
    
    \large Questions?
\end{frame}

\end{document}